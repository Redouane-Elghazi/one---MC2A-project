\documentclass[twocolumn]{article}
%\documentclass{article}
\usepackage{lipsum}
\usepackage[french]{babel}
\usepackage[T1]{fontenc}
\usepackage[utf8]{inputenc}
\usepackage{hyperref}
\usepackage{setspace}
\usepackage{multicol}
\usepackage{amsfonts}
\usepackage{amsmath}

\doublespacing
\date{}
\author{Redouane ELGHAZI \and Pierre MAHMOUD--LAMY \and Enguerrand PREBET}
\title{Projet de MC2A : Équipe one}
\begin{document}
	\maketitle
	\section{Introduction}
		Le but de ce projet était d'implémenter une méthode de Monte-Carlo par chaînes de Markov appelée algorithme de Metropolis-Hastings. Cet algorithme a pour entrée un ensemble $P$ de points de $\mathbb{R}^N$ et une fonction $\emph{label}^*$ associant un label à chaque point.
		
		Le but de l'algorithme est de trouver une fonction $\emph{label}$ minimisant le nombre de mauvais labels. Dans le cadre de ce projet, la fonction recherchée est de la forme:
		$$\emph{label}(x) = \text{sign}(w\cdot x)$$
		Où $w$ est un vecteur de $\big\{0,1\big\}^N$.
		 
		Pour se faire, à chaque étape, un bit de $w$ est proposé à la modification, et est accepté avec un probabilité dépendant du nombre de points mal classifiés avant et après modification.
	\section{Langage}
		C++, Python
	\section{Résultats}
		analyse de la complexité de l'algo
		
		oui des plot (q1)
	\section{Analyse de la valeur de $\beta$}
		oui des plot c'est cool (q2-3)
		puis replot de la q1 avec des beta hinted
	\section{Qu'apporte le simulated annealing}
		encore des plot oui oui des plot
		
	\section{Conclusion}
		Je remercie mon manager sans qui rien n'aurait été possible ah non c'est pas les remerciements
\end{document}